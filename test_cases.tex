\documentclass{sener2025}
\usepackage{booktabs}
\usepackage{colortbl}
\usepackage{tabularx}
\usepackage{xltabular}
\usepackage{float}
\usepackage{ragged2e}

\begin{document}

\section{Pruebas de Formato de Tablas y Notas}

A continuación se presentan los casos de prueba para verificar la corrección de:
\begin{itemize}
    \item Alineación de fuentes y notas al pie.
    \item Eliminación de saltos de línea incorrectos en fuentes.
    \item Ordenamiento y formato de notas con asteriscos (*, **, ***).
    \item Uso de notas al pie numeradas dentro de tablas.
\end{itemize}

\subsection{Caso 1: Notas con Asteriscos (*, **, ***)}
Esta tabla muestra notas extraídas automáticamente y ordenadas por jerarquía.

\begin{tabladoradoCorto}
  \caption{Tabla con Notas de Asteriscos}
  \label{tab:asteriscos}
  \begin{tabularx}{\textwidth}{Vv}
    \toprule
    \rowcolor{gobmxDorado} \encabezadodorado{Concepto} & \encabezadodorado{Valor} \\
    \midrule
    Dato A * & 100 \\
    Dato B ** & 200 \\
    Dato C *** & 300 \\
    \bottomrule
  \end{tabularx}
\end{tabladoradoCorto}
\vspace{-4pt}
{\raggedright\notosanslight\fontsize{7pt}{9pt}\selectfont\setlength{\parskip}{0pt}
* Nota de primer nivel (un asterisco). \par
** Nota de segundo nivel (dos asteriscos). \par
*** Nota de tercer nivel (tres asteriscos). \par
\vspace{1pt}
{\color{gobmxGris} FUENTE:~Elaboración propia con datos de prueba.}\hfill
\par}
\vspace{6pt}

\subsection{Caso 2: Fuente con Notas al Pie Numeradas (Solución Inline)}
Este caso verifica que los saltos de línea en el campo "Fuente" (que causaban que los números 10 y 11 aparecieran en nuevas líneas) han sido eliminados.
Se simula el uso de \texttt{[[nota:...]]} o entrada manual.

\begin{tabladoradoCorto}
  \caption{Tabla con Fuente y Notas al Pie}
  \label{tab:footnotes}
  \begin{tabularx}{\textwidth}{Vv}
    \toprule
    \rowcolor{gobmxDorado} \encabezadodorado{Concepto} & \encabezadodorado{Referencia} \\
    \midrule
    Infraestructura & Ver nota 10 \\
    Prontuario & Ver nota 11 \\
    \bottomrule
  \end{tabularx}
\end{tabladoradoCorto}
\vspace{-4pt}
% Aquí se muestra el resultado esperado: todo en una línea o con footnotes reales
\fuente{Elaboración de SENER con información de Infraestructura Nacional... \footnote{Infraestructura Nacional de Almacenamiento...} \footnote{Prontuario Estadístico de la SENER.}}

\subsection{Caso 3: Verificación de Limpieza de Newlines}
Si el usuario ingresa saltos de línea manuales entre marcadores:

\begin{tabladoradoCorto}
  \caption{Tabla con Saltos de Línea en Fuente}
  \label{tab:newlines}
  \begin{tabularx}{\textwidth}{Vv}
    \toprule
    \rowcolor{gobmxDorado} \encabezadodorado{Item} & \encabezadodorado{Detalle} \\
    \midrule
    Prueba & Texto \\
    \bottomrule
  \end{tabularx}
\end{tabladoradoCorto}
\vspace{-4pt}
% El generador ahora convierte los saltos en espacios:
\fuente{Elaboración propia.   10   11}

\end{document}
