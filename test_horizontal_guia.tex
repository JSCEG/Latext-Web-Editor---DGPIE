\documentclass{sener2025}
\usepackage{lipsum}

\begin{document}

\section{Página Vertical Inicial}
Esta es una página vertical normal con su encabezado y pie de página.
\lipsum[1]

\begin{figuraespecial}
  % Título opcional para evitar orfandad en página anterior
  \tituloHorizontal{Prueba de Imagen como Sección Horizontal con Nota al Pie\footnote{Nota al pie en el título para verificar posición.}}

  % Contenido activo para prueba final
  \captionHorizontal{Evolución del consumo energético nacional durante el periodo 2020-2024. Este texto debe aparecer alineado a la izquierda.}
  \imagenHorizontal{img/figura_6_6.png}{fig:test}
  \fuenteHorizontal{Fuente: SENER, Balance Nacional de Energía 2024. Elaboración propia con datos del Sistema de Información Energética.\footnotemark}
  \footnotetext{Esta es una nota al pie de prueba para verificar su comportamiento en modo horizontal.}
  
  %\vspace*{\fill}
\end{figuraespecial}

\begin{figuraespecial}
  % Título opcional
  \subseccionHorizontal{Prueba de Subsección Numerada (Debe ser 0.1)}
  \sinNotas % Maximizar altura al no tener notas al pie

  % Contenido activo
  \captionHorizontal{Misma imagen pero maximizada al no tener nota al pie en la fuente.}
  \imagenHorizontal{img/figura_6_6.png}{fig:test2}
  \fuenteHorizontal{Fuente: SENER, Balance Nacional de Energía 2024. Elaboración propia con datos del Sistema de Información Energética.}
\end{figuraespecial}

\section{Regreso a Página Vertical}
Esta página debería tener nuevamente el encabezado y pie de página verticales normales, incluyendo el número de página.
\lipsum[1-5]

\end{document}
